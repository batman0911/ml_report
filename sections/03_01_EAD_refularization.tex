\section{Sơ bộ về hiệu chỉnh Elastic-Net}
Hiệu chỉnh elastic-net là kỹ thuật được sử dụng rộng rãi trong việc giải quyết 
các bài toán lựa chọn thuộc tính nhiều chiều (Zou and Hastie 2005). Nó 
là hiệu chỉnh sử dụng tổ hợp tuyến tính hàm phạt (\textit{penalty})
$L_1$ và $L_2$. Nhìn chung, hiệu chỉnh elastic-net được sử dụng trong bài toán cực 
tiểu hóa sau đây:
\begin{equation}
    \label{eq:3}
    \text{minimize}_{\mathbf{z} \in \mathcal{Z}} \text{ }
    f(\mathbf{z}) + \lambda_1 \lVert \mathbf{z} \rVert_1
    + \lambda_2 \lVert \mathbf{z} \rVert_2^2
\end{equation}
Trong đó $\mathbf{z}$ là vector của $p$ biến tối ưu, $\mathcal{Z}$ là tập nghiệm 
chấp nhận được, $f(\mathbf{z})$ là hàm mất mát, $\lVert \mathbf{z} \rVert_q$ là 
chuẩn $q$ của $\mathbf{z}$ và $\lambda_1, \lambda_2 \geq 0$ tương ứng là các tham số hiệu 
chỉnh $L_1$ và $L_2$. Biểu thức $\lambda_1 \lVert \mathbf{z} \rVert_1 + \lambda_2 
\lVert \mathbf{z} \rVert_2^2$ được gọi là hiệu chỉnh elactic-net của $\mathbf{z}$.
Với bài toán hồi quy chuẩn, hàm mất mát $f(\mathbf{z})$ là hàm sai số bình phương trung bình
(\textit{mean squared error - MSE}), vector $\mathbf{z}$ biểu diễn trọng 
số của các thuộc tính và $\mathcal{Z} = \mathbb{R}^p$. Hiệu chỉnh elastic-net trong 
phương trình \ref{eq:3} chính là công thức Lasso khi $\lambda_2 = 0$ và trở thành công 
thức hồi quy Ridge khi $\lambda_1 = 0$. (Zou and Hastie 2005) đã chỉ ra rằng hiệu chỉnh 
elastic-net có thể chọn ra 1 nhóm các thuộc tính tương quan mạnh, khắc phục nhược điểm của việc chọn lựa thuộc tính nhiều chiều khi sử dụng 1 mình hồi quy Lasso 
hoặc hồi quy Ridge.