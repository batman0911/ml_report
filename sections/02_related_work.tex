\chapter{Các nghiên cứu liên quan}
Trong phần này, tác giả tổng hợp các nghiên cứu liên quan về tấn công và phòng thủ DNNs.

\section{Tấn công vào DNNs}
\underline{FGM và I-FGM:} Kí hiệu $\mathbf{x_0}$ và $\mathbf{x}$ lần lượt là mẫu gốc và mẫu đối nghịch,
$t$ là lớp mục tiêu cần tấn công. Các phương pháp gradient nhanh (\textit{fast gradient 
methods - FGM}) sử dụng gradient $\nabla J$ của hàm mất mát $J$ tại $\mathbf{x_0}$
để tạo ra các mẫu đối nghịch (Goodfellow, Shlens, and Szegedy 2015). Với tấn công $L_{\infty}$, 
$\mathbf{x}$ được tính bằng công thức:
\begin{equation}
    \mathbf{x} = \mathbf{x_0} - \epsilon \times \text{sign}(\nabla J(\mathbf{x_0}, t))
    \label{eq:1}
\end{equation}
Trong đó $\epsilon$ là độ biến dạng $L_{\infty}$ giữa $\mathbf{x}$ và $\mathbf{x_0}$ và 
$\text{sign}(\nabla J)$ trả về dấu của gradient. Với tấn công $L_1$ và $L_2$, $\mathbf{x}$ 
được tính bằng công thức:
\begin{equation}
    \mathbf{x} = \mathbf{x_0} - \epsilon \frac{\nabla J(\mathbf{x_0}, t)}
    {\lVert \nabla J(\mathbf{x_0}, t) \rVert _q}
    \label{eq:2}
\end{equation}
với $q = 1,2$ và $\epsilon$ là độ nhiễu tương ứng. Các phương pháp lặp gradient nhanh 
(\textit{Iterative fast gradient methods (I-FGM)}) được trình bày trong (Kurakin, Goodfellow, 
and Bengio 2016b) là phương pháp lặp sửa dụng FGM với một độ nhiễu mịn hơn. Tấn công 
không nhắm đích sử dụng FGM và I-FGM cũng được triển khai theo cách tương tự. 

\underline{Tấn công C\&W:} Thay vì sử dụng hàm mất mát trên tập huấn luyện Carlini và Wagner
đã thiết kế  một hiệu chỉnh $L_2$ trong hàm mất mát dựa trên lớp logit trong DNNs để sinh 
ra các mẫu đối nghịch (Carlini and Wagner 2017b). Công thức này hóa ra là một trường hợp riêng của 
thuật toát EAD (sẽ được trình bày trong phần sau). Tấn công C\&W được coi 
là một trong những tấn công mạnh nhất đối với DNNs vì nó có thể phá vỡ cả những DNN không phòng thủ và DNN phòng thủ chắt lọc, nó cũng có khả năng chuyển giao tấn công khá tốt. 

\underline{DeepFool:} là thuật toán tấn công $L_2$ không nhắm đích (Moosavi-Dezfooli, 
Fawzi, and Frossard 2016) dựa trên lý thuyết về phép chiếu tới siêu phẳng phân tách gần nhất
trong bài toán phân lớp. Nó cũng được sử dụng để tạo ra một nhiễu loạn phổ quát nhằm đánh lừa 
các DNN vốn đã được huấn luyện trên các ảnh gốc (Moosavi-Dezfooli et al. 2016).

\section{Phòng thủ trong DNNs}
\underline{Chắt lọc phòng thủ (Defensive distillation):} Chắt lọc phòng thủ (Papernot et al.2016b) 
chống lại nhiễu loạn đối nghịch bằng cách sử dụng kỹ thuật chắt lọc trong (Hinton, 
Vinyals, and Dean 2015) để huấn luyện lại cùng một mạng với xác suất lớp được dự đoán bởi 
mạng ban đầu. Phương pháp này đưa ra tham số nhiệt độ $T$ trong lớp softmax để tăng cường 
sức mạnh của mạng trước những nhiễu loạn đối nghịch.

\underline{Huấn luyện đối nghịch (Advrsarial training):} có thể được triển khai theo 1 số cách khác nhau. 
Hướng tiếp cận chuẩn là thêm vào tập dữ liệu huấn luyện ban đầu các mẫu đối nghịch đã được 
sửa đúng nhãn, sau đó huấn luyện lại mạng. (Zheng et al. 2016; Madry et al. 2017; 
Tram`er et al. 2017; Zantedeschi, Nicolae, and Rawat 2017) đề xuất điều chỉnh training 
loss hoặc kiến trúc mạng để tăng cường sức mạnh của DNNs trước mẫu đối nghịch.

\underline{Các phương pháp phát hiện:} Sử dụng các test thống kê để phân biệt các mẫu đối nghịch 
với mẫu gốc (Feinman et al. 2017; Grosse et al. 2017; Lu, Issaranon, and Forsyth 2017; 
Xu, Evans, and Qi 2017). Tuy nhiên, 10 phương pháp phát hiện khác nhau đã không thể phát 
hiện được tấn công C\&W (Carlini and Wagner 2017a).