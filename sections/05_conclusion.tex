\chapter{Kết luận}
Nhóm tác giả đã đề xuất mô hình tấn công bằng hiệu chỉnh elastic-net để tạo ra các mẫu đối nghịch trong tấn công DNN. Các kết quả thực nghiệm trên các tập dữ liệu MNIST, CIFAR10 và ImageNet cho thấy các mẫu $L_1$ tạo bởi EAD có thể đạt được tỷ lệ thành công tương đương với các phương pháp tấn công tiên tiến dựa trên $L_2$ và $L_{\infty}$ khi phá vỡ mạng không phòng thủ và phòng thủ chưng cất. Ngoài ra, EAD có thể cải thiện khả năng chuyển giao tấn công và huấn luyện đối nghịch bổ sung. Các kết quả của nhóm tác giả đã chứng minh hiệu quả của EAD và đưa ra hướng mới sử dụng mẫu đối nghịch $L_1$ trong việc huấn luyện đối nghịch và tăng cường an ninh cho DNN.