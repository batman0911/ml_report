\section{Sơ bộ về hiệu chỉnh Elastic-Net}
Hiệu chỉnh elastic-net là công nghệ được sử dụng rộng rãi trong việc giải quyết 
các bài toán lựa chọn thuộc tính nhiều chiều (Zou and Hastie 2005). Nó được xem 
là công cụ hiệu chỉnh trong đó tổ hợp tuyến tính hàm phạt (\textit{penalty})
$L_1$ và $L_2$. Nhìn chung, hiệu chỉnh elastic-net được sử dụng trong bài toán cực 
tiểu hóa sau đây:
\begin{equation}
    \text{minimize}_{\mathbf{z} \in \mathcal{Z}} \text{ }
    f(\mathbf{z}) + \lambda_1 \lVert \mathbf{z} \rVert_1
    + \lambda_2 \lVert \mathbf{z} \rVert_2^2
\end{equation}
Trong đó $\mathbf{z}$ là véc tơ của $p$ biến tối ưu, $\mathcal{Z}$ là tập nghiệm 
chấp nhận được, $f(\mathbf{z})$ là hàm mất mát, $\lVert \mathbf{z} \rVert_q$ là 
chuẩn $q$ của $\mathbf{z}$ và $\lambda_1, \lambda_2 \geq 0$ tương ứng là các tham số hiệu 
chỉnh $L_1$ và $L_2$. Biểu thức $\lambda_1 \lVert \mathbf{z} \rVert_1 + \lambda_2 
\lVert \mathbf{z} \rVert_2^2$ là được gọi là hiệu chỉnh elactic-net của $\mathbf{z}$.
Với bài toán hồi quy chuẩn, hàm mất mát $f(\mathbf{z})$ là hàm lỗi trung trung bình
bình phương (\textit{mean squared error - MSE}), véc tơ $\mathbf{z}$ biểu diễn trọng 
số của các thuộc tính và $\mathcal{Z} = \mathbb{R}^p$