\chapter{Tổng quan}
Xem xét hệ phương trình tuyến tính $Ax=b$ trong đó $A \in \mathbb{C} ^{m \times m}$ 
và $b \in \mathbb{C}^m$.

Nghiệm chính xác của hệ $x_* = A^{-1}b$. Đương nhiên việc tính ma trận nghịch đảo $A^{-1}$
hoặc dùng phương pháp khử Gauss là rất tốn chi phí. Ta biết rằng độ phức tạp của thuật 
toán khi sử dụng phương pháp khủ Gauss là $O(m^3)$. Để có cái nhìn trực quan về độ phức tạp 
này, ta có thể xem qua bảng sau 

\begin{center}
    \begin{tabular}{| c | c | c |}
        \hline
        Năm & $m$ & $m^3$ \\ 
        \hline
        1950 & 20 & $2 \times 10^3$ \\
        1965 & 200 & $2 \times 10^6$ \\
        1980 & 2000 & $2 \times 10^9$ \\
        \hline
    \end{tabular}
\end{center}

Khi kích thước của ma trận tăng lên, số phép tính cũng tăng lên rất nhanh. Cùng với sự phát triển của 
phần cứng chúng ta cũng tính toán được với những ma trận với kích thước lớn hơn rất nhiều so với những
năm 1950. Nhưng ta cũng thấy rằng, khi kích thước của ma trận là hàng nghìn thì số phép tính đã là 
hàng tỉ. Với những bài toán lớn như mô phỏng thời tiết hay thiên hà trong thiên văn học, phần cứng 
máy tính không thể đuổi kịp để tính toán cũng như lưu trữ.

Điều này đòi hỏi chúng ta cần tìm ra những phương pháp tốt hơn nhằm giảm độ phức tạp của thuật toán.
Tiểu luận này giới thiệu 1 vài phương pháp với độ phức tạp $O(m^2)$.

Phương pháp lặp cho phép chúng ta tiến gần đến nghiệm chính xác của phương trình qua mỗi bước lặp.
Đến một lúc nào đó phần dư là đủ chấp nhận được, ta có nghiệm gần đúng của phương trình.

Gọi $x_n$ là nghiệm gần đúng của phương trình tại bước lặp thứ $n$. Ta định nghĩa phần dư
\begin{equation}
    r_n = b - Ax_n
\end{equation}

Khi $r_n < \epsilon$ (là một sai số nhỏ chấp nhận được) ta dừng lại và thu được nghiệm gần đúng $x_n$

Trong bài này chúng ta sẽ tìm nghiệm $x_n \in \kappa_n$, trong đó $\kappa_n$ là không gian con Krylov

\section*{Không gian con Krylov}
Cho ma trận $A \in \mathbb{C}^{m \times m}$ và vector $b \in \mathbb{C}^m$.

Dãy Krylov được định nghĩa là tập các vectors $b, Ab, A^2b, ...$.

Không gian con Krylov là không gian được sinh bởi các vectors này

Ma trận Krylov
\begin{equation}
    K_n = \begin{bmatrix}
        b & | & Ab & | & A^2b & ... & | & A^{n-1}b
    \end{bmatrix}    
\end{equation}


Trở lại bài toán, ta muốn tìm nghiệm $x_n$ trong không gian $\kappa_n$ hay nói 
các khác $x_n \in range(K_n)$. Ta có thể sử dụng phân tích $QR$ cho ma trận Krylov và
đặt $x_n = Q_ny$ trong đó $Q_n$ là ma trận unitary. 

Bài toán cực tiểu hóa $r_n$ đưa về việc cực tiểu hóa $||AQ_ny - b||$


Chúng ta sẽ tiếp cận bài toán với việc giải tổng quát với phương pháp GMRES 
và sử giải bài toán ma trận đối xứng như một trường hợp riêng với phương pháp MINRES.
Trước hết, thuật giải Arnoldi sẽ được giới thiệu sau đây.