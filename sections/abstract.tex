\begin{abstract}
    Các nghiên cứu gần đấy đã chỉ ra tính dễ bị tổn thương của các mạng nơ ron sâu 
    (Deep Neural Networks - DNNs) đến các mẫu đối nghịch - một cách trực quan, hình ảnh
    đối nghịch không thể phân biệt có thể được tạo ra một cách dễ dàng khiến các mô hình 
    được huấn luyện tốt cũng phân loại ảnh sai. Các phương pháp hiện có để tạo ra mẫu 
    đối nghịch thường dựa trên thông số biến dạng $L_2$ và $L_{\infty}$. Mặc dù trong 
    thực tế độ biến dạng $L_1$ là quan trọng và quyết định đến tính thưa của nhiễu lại 
    ít được xem xét. 
    Trong bài này, chúng tôi thiết kế một quy trình tấn công DNNs thông qua mẫu đối nghịch
    như một bài toán tối ưu hóa sử dụng hiệu chỉnh elastic-net. Tấn công DNNs bằng 
    elastic-net (EAD) với tham số $L_1$ được thêm vào cùng với tấn công $L_2$. 
    Kết quả thực nghiệm trên các tập dữ liệu MNIST, CIFAR10 và ImageNet chỉ ra rằng
    EAD có thể mang lại một tập mẫu đối nghịch với độ nhiều $L_1$ nhỏ và đạt được hiệu 
    suất tấn công tương đương với các phương pháp hiện đại nhất qua các kịch bản tấn 
    công khác nhau. Quan trọng hơn, EAD dẫn đến sự cải tiến trong việc tấn công DNN và
    gợi ý những hiểu biết mới về hiệu chỉnh $L_1$ để cải thiện bảo mật trong các mô 
    hình học máy.
\end{abstract} 