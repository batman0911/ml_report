\section{Thuật toán EAD}
Khi giải bài toán EAD trong \ref{eq:3.5} mà không có thành phần $L_1$, Carlini and Wagner 
sử dụng kĩ thuật COV (change-of-variable) qua hàm \textit{tanh} trên $\mathbf{x}$ nhằm 
mục đích loại bỏ ràng buộc $\mathbf{x} \in [0,1]^p$ (Carlini and Wagner 2017b). Khi 
$\beta > 0$, ta thấy COV sẽ không hiệu quả để giải bài toán \ref{eq:3.5} vì mẫu đối 
nghịch tương ứng sẽ không nhạy cảm với thay đổi trên $\beta$ (xem phần đánh giá hiệu 
năng để biết chi tiết). Vì $L_1$ không khả vi trên toàn không gian, COV không giải 
được bài toán \ref{eq:3.5} do không dùng được phương pháp dưới đạo hàm (\textit{subgradient})
trong việc giải bài toán tối ưu (Duchi and Singer 2009). \\

Để giải bài toán EAD trong \ref{eq:3.5} để sinh ra các mẫu đối nghịch, bài viết này đề
xuất sử dụng thuật toán ISTA (\textit{iterative shrinkage-thresholding algorithm – thuật toán lặp 
với ngưỡng biến đổi}) (Beck and Teboulle 2009). STA được coi như thuật toán tối ưu bậc nhất
phổ biến với thêm một bước điều chỉnh ngưỡng trong mỗi bước lặp. Cụ thể, 
đặt $g(\mathbf{x}) = c \times f(\mathbf{x}) + \lVert \mathbf{x} - \mathbf{x_0} \rVert_2^2$
và đặt $\nabla g(\mathbf{x})$ là gradient của $g(\mathbf{x})$ được tính bơi mạng DNN. 
Tại bước lặp thứ $k+1$, mẫu đối nghịch $\mathbf{x}^{(k+1)}$ của $\mathbf{x_0}$ được tính 
như sau:
\begin{equation}
    \label{eq:3.6}
    \mathbf{x}^{(k+1)} = S_{\beta} (\mathbf{x}^{(k)} - \alpha_k \nabla g(\mathbf{x}^{(k)}))
\end{equation}
Trong đó, $\alpha_k$ là độ dài bước tại bước lặp thứ $k+1$, $S_{\beta} : \mathbb{R}^p \to 
\mathbb{R}^p$ là hàm của phép chiếu biến đổi ngưỡng trên từng phần tử, được xác định bởi:
\begin{equation}
    \label{eq:3.7}
    [S_{\beta}(\mathbf{z})]_i = 
    \begin{cases}
        \min \{ \mathbf{z}_i - \beta, 1 \} &\text{ nếu } \mathbf{z}_i - \mathbf{x_0}_i  > \beta; \\
        \mathbf{x_0}_i &\text{ nếu } |\mathbf{z}_i - \mathbf{x_0}_i| \leq \beta; \\
        \max \{ \mathbf{z}_i + \beta, 0 \} &\text{ nếu } \mathbf{z}_i - \mathbf{x_0}_i < -\beta
    \end{cases}
\end{equation}
Với $i \in \{ 1, ..., p \}$. Nếu $|\mathbf{z}_i - \mathbf{x_0}_i| > \beta$, thành phần 
$\mathbf{z}_i$ được co lại với hệ số $\beta$ và chiếu thành phần kết quả lên miền ràng buộc 
chấp nhận được thuộc đoạn $[0,1]$. Mặt khác, nếu 