\section{Phân tích độ nhạy và luật quyết định}
Tác giả kiểm tra sự cần thiết của việc sử dụng thuật toán 1 để giải bài toán tấn công bằng hiệu chỉnh elastic-net trong (\ref{eq:7}) bằng cách so sánh nó với thuật toán COV thuần. Trong tài liệu (Carlini and Wagner 2017b), Carlini và Wagner đã loại bỏ điều kiện ràng buộc $\mathbf{x} \in [0,1]^p$ bằng cách thay $\mathbf{x}$ bằng $\frac{\mathbf{1} + \tanh \mathbf{w}}{2}$ với $\mathbf{w} \in \mathbb{R}^p$ và $\mathbf{1} \in \mathbb{R}^p$ là vector các với các phần tử $1$. Thuật toán tối ưu mặc định ADAM (Kingma and Ba 2014) được sử dụng để giải ra nghiệm $\mathbf{w}$ và từ đó tìm được $\mathbf{x}$. Tác giả áp dụng thuật toán COV lên (\ref{eq:7}) và so sánh với EAD trên tập MNIST với các  khác nhau trong bảng 1 ở trên. Mặc dù COV và EAD thu được tỷ lệ tấn công thành công tương tự nhau, người ta quan sát thấy COV không hiệu quả trong việc tạo ra các mẫu đối nghịch $L_1$. Khi tăng $\beta$, EAD tạo ra ít mẫu đối nghịch $L_1$ hơn, trong khi các kết quả $L_1$, $L_2$, và $L_{\infty}$ của COV thì ít chịu ảnh hưởng khi thay đổi $\beta$. Khi sử dụng các hàm thư viện TensorFlow như AdaGrad, RMSProp, SGD, người ta cũng thấy nó ít ảnh hưởng bởi $\beta$  giống như trường hợp của COV. Nhóm tác giả cũng báo cáo rằng phương pháp COV cấm dùng ISTA vì thành phần hàm $\tanh$ trong hiệu chỉnh $L_1$. Sự ít ảnh hưởng của COV cho thấy nó không phù hợp cho tối ưu elastic-net, do nó không hiệu quả cho bài toán tối ưu bằng subgradient (Duchi and Singer 2009). Với EAD, nhóm tác giả tìm được 1 cách thú vị đánh đổi giữa $L_1$, $L_2$, và $L_{\infty}$. Điều này là do khi tăng $\beta$ làm tăng độ thưa của nhiễu và do đó tăng $L_2$, $L_{\infty}$ . Kết quả tương tự cũng được quan sát thấy với CIFAR10.
\begin{figure}[H] % places figure environment here   
    \centering % Centers Graphic
    \includegraphics[width=1\textwidth]{assets/fig_02.png} 
    \caption{So sánh luật quyết định EN và $L_1$ trong EAD trên tập MNIST với nhiều tham số hiệu chỉnh $L_1$ $\beta$ (trường hợp trung bình). So sánh với luật EN cho cùng giá trị $\beta$} % Creates caption underneath graph
    \label{fig:fg_02}
\end{figure}

\begin{longtable}{l}
		\resizebox{\textwidth}{!}{
		\begin{tabular}{l|llll|llll|llll}
			\hline
			& \multicolumn{4}{c}{MNIST} & \multicolumn{4}{c}{CIFAR 10} & \multicolumn{4}{c}{ImageNet}\\
			\hline
			Attack method & ASR & $L_1$ & $L_2$ & $L_\infty$ & ASR & $L_1$ & $L_2$ & $L_\infty$ & ASR & $L_1$ & $L_2$ & $L_\infty$ \\
			\hline
			C\&W ($L_2$) & \textbf{100} & 22.46 & \textbf{1.972} & 0.514 & \textbf{100} & 13.62 & \textbf{0.392} & 0.044 & \textbf{100} & 232.2 & \textbf{0.705} & 0.03 \\
			FGM-$L_1$ & 39 & 53.5 & 4.186 & 0.782 & 48.8 & 51.97 & 1.48 & 0.152 & 1 & 61 & 0.187 & 0.007 \\
			FGM-$L_2$ & 34.6 & 39.15 & 3.284 & 0.747 & 42.8 & 39.5 & 1.157 & 0.136 & 1 & 2338 & 6.823 & 0.25 \\
			FGM-$L_\infty$ & 42.5 & 127.2 & 6.09 & 0.296 & 52.3 & 127.81 & 2.373 & 0.047 & 3 & 3655 & 7.102 & 0.014 \\
			I-FGM-$L_1$ & \textbf{100} & 32.94 & 2.606 & 0.591 & \textbf{100} & 17.53 & 0.502 & 0.055 & 77 & 526.4 & 1.609 & 0.054 \\
			I-FGM-$L_2$ & \textbf{100} & 30.32 & 2.41 & 0.561 & \textbf{100} & 17.12 & 0.489 & 0.054 & \textbf{100} & 774.1 & 2.358 & 0.086 \\
			I-FGM-$L_\infty$ & \textbf{100} & 71.39 & 3.472 & \textbf{0.227} & \textbf{100} & 33.3 & 0.68 & \textbf{0.018} & \textbf{100} & 864.2 & 2.079 & \textbf{0.01} \\
			EAD (EN rule) & \textbf{100} & \textbf{17.4} & 2.001 & 0.594 & \textbf{100} & \textbf{8.18} & 0.502 & 0.097 & \textbf{100} & \textbf{69.47} & 1.563 & 0.238 \\
			EAD ($L_1$ rule) & \textbf{100} & \textbf{14.11} & 2.211 & 0.768 & \textbf{100} & \textbf{6.066} & 0.613 & 0.17 & \textbf{100} & \textbf{40.9} & 1.598 & 0.293 \\
			\hline
		\end{tabular}} \\ 
		\caption{So sánh các tấn công trên các tập dữ liệu MNIST, CIFAR10 và ImageNet. ASR là tỷ lệ tấn công thành công (\%). Giá trị nhiễu trong bảng đo được trên giá trị trung bình của các mẫu thành công. EAD, C\&W và I-FGM-$L_\infty$ thu được các mẫu ít nhiễu nhất trên các chuẩn $L_1$,$L_2$,$L_\infty$ tương ứng. Kết quả đầy đủ được báo cáo trong tài liệu mở rộng.}
		\label{tab:tab_2}
\end{longtable}

Ở bảng \ref{tab:tab_1}, quá trình thuật toán tối ưu chạy để tìm ra mẫu đối nghịch cuối cùng giữa những mẫu đối nghịch thành công dựa trên hàm mất mát elastic-net trong $\{\mathbf{x}^{(k)}\}^I_{k=1}$, gọi là luật chọn elastic-net. Ngoài ra, có thể chọn mẫu đối nghịch cuối cùng sao cho $L_1$ nhỏ nhất, gọi là luật chọn $L_1$. Hình \ref{fig:fg_02} so sánh ASR và các nhiễu trung bình trên 2 luật chọn nói trên với các giá trị  khác nhau trên tập dữ liệu MNIST. Cả 2 luật chọn đều cho tỷ lệ thành công ASR $100\%$ với dải giá trị  rộng. Với cùng 1 giá trị , luật chọn $L_1$ cho ra các mẫu đối nghịch ít nhiễu $L_1$ hơn so với luật chọn EN, trong khi đó lại bị trả giá nhiều nhiễu hơn ở $L_2$, $L_{\infty}$ . Xu hướng tương tự cũng được quan sát thấy với CIFAR10 (kết quả đầy đủ với cả 2 tập dữ liệu MNIST và CIFAR có trong tài liệu mở rộng). Trong các thí nghiệm tiếp theo, nhóm tác giả báo cáo kết quả của EAD với 2 luật chọn và $\beta = 10^{-3}$, vì trên tập MNIST và CIFAR10 giá trị $\beta$ làm giảm nhiễu $L_1$ trong khi so sánh với $L_2$, $L_{\infty}$ với trường hợp $\beta = 0$ (nghĩa là không có hiệu chỉnh $L_1$).

