\section{Xây dựng thuật toán EAD}

Từ ý tưởng của tấn công C\&W (Carlini and Wagner 2017b), ta xem xét cùng hàm mất mát $f$ để sinh 
ra các mẫu đối nghịch. Cụ thể, cho trước 1 ảnh $\mathbf{x_0}$ với nhãn đúng của nó là $t_0$, 
gọi $\mathbf{x}$ là mẫu đối nghịch của $\mathbf{x_0}$ với lớp đích nhắm đến là $t \neq t_0$ . Hàm mất mát 
$f(\mathbf{x})$ cho tấn công nhắm đích là:
\begin{equation}
    \label{eq:4}
    f(\mathbf{x}, t) = \max { \left( \max_{j \neq t} [\textbf{Logit}(\mathbf{x})]_j - 
    [\textbf{Logit}(\mathbf{x})]_t, -\kappa \right) }
\end{equation}
Trong đó $\textbf{Logit}(\mathbf{x}) = [[\textbf{Logit}(\mathbf{x})]_1, ..., 
[\textbf{Logit}(\mathbf{x})]_K] 
\in \mathbb{R}^K$ là lớp logit (lớp trước softmax) biểu diễn cho $\mathbf{x}$ trong mạng DNN, $K$
là số lượng lớp cần phân loại, $\kappa > 0$ là tham số độ tin cậy, nó đảm bảo một khoảng 
cách cố định giữa $\max_{j \neq t} [\textbf{Logit}(\mathbf{x})]_j$ và $[\textbf{Logit}(\mathbf{x})]_t$. 

Cần lưu ý rằng, thành phần $[\textbf{Logit}(x)]_t$ là xác xuất dự đoán $x$ có nhãn $t$ theo 
luật phân loại của hàm softmax:
\begin{equation}
    \label{eq:5}
    \text{Prob}(\text{Label}(\mathbf{x}) = t) = \frac{\exp([\textbf{Logit}(\mathbf{x})]_t)}{
        \sum_{j=1}^{K} \exp([\textbf{Logit}(\mathbf{x})]_j)
    }
\end{equation}
Do đó, hàm mất mát trong \ref{eq:4} có mục đích là cho ra nhãn $t$ là 
lớp có xác xuất cao nhất của $x$ và tham số $\kappa$ đảm bảo sự phân biệt giữa lớp $t$
và lớp dự đoán gần nhất khác với $t$. Với tấn công không nhắm đích, hàm mất mát trong 
\ref{eq:4} trở thành:
\begin{equation}
    \label{eq:6}\
    f(\mathbf{x}) = \max { \left([\textbf{Logit}(\mathbf{x})]_{t_0} - 
    \max_{j \neq t} [\textbf{Logit}(\mathbf{x})]_j, -\kappa \right) }
\end{equation}
Trong bài viết này, tác giả tập trung vào tấn công nhắm đích vì nó khó hơn so với tấn công không 
nhắm đích. Thuật toán EAD có thể được áp dụng trực tiếp vào tấn công không nhắm đích bằng cách thay $f(\mathbf{x},t)$ trong \ref{eq:4} thành $f(\mathbf{x})$ trong \ref{eq:6}. 

Ngoài ra, để thao túng kết quả dự đoán thông qua hàm mất mát trong \ref{eq:4}, hiệu chỉnh 
elastic-net còn tạo ra mẫu đối nghịch tương tự với ảnh gốc. Công thức tấn công elastic-net
vào mạng DNNs (EAD) để tạo ra mẫu đối nghịch $(\mathbf{x},t)$ cho ảnh gốc $(\mathbf{x_0}, t_0)$ như sau:
\begin{equation}
    \label{eq:7}
    \begin{split}
    &\text{minimize}_{\mathbf{x}} \text{ }
    c \times f(\mathbf{x}, t) + \beta \lVert \mathbf{x} - \mathbf{x_0} \rVert_1
    + \lVert \mathbf{x} - \mathbf{x_0} \rVert_2^2 \\
    &\text{st   } \mathbf{x} \in [0,1]^p
    \end{split}
\end{equation}
Với $f(\mathbf{x},t)$ được xác định trong \ref{eq:4}, $c, \beta \geq 0$ lần lượt 
là các tham số  hiệu chỉnh của hàm mất mát $f$ và hàm phạt $L_1$. Miền ràng buộc 
$\mathbf{x} \in [0,1]^p$ giữ cho $\mathbf{x}$ trong một không gian ảnh mà có thể dễ dàng thỏa mãn điều kiện
bằng cách chia giá trị của mỗi điểm ảnh cho giá trị lớn nhất (ví dụ $255$). Để xác định khoảng 
cách (nhiễu) giữa $\mathbf{x}$ và $\mathbf{x_0}$ bằng $\mathbf{\delta} = \mathbf{x} - \mathbf{x_0}$,
công thức EAD trong \ref{eq:7} được tạo ra để tìm một mẫu đối nghịch $\mathbf{x}$ mà sẽ 
được phân loại vào lớp mục tiêu $t$, đồng thời cực tiểu hóa $\mathbf{\delta}$ trong hàm
mất mát elastic-net $\beta \lVert \mathbf{\delta} \rVert_1 + \lVert \mathbf{\delta} \rVert_2^2$.
Đáng chú ý, công thức tấn công C\&W (Carlini and Wagner 2017b) trở thành trường hợp đặc biệt 
của EAD trong công thức \ref{eq:7} khi $\beta = 0$, không phụ thuộc vào $L_1$ của $\mathbf{\delta}$.
Tuy nhiên thành phần phạt $L_1$ là một hiệu chỉnh trực quan để tạo ra mẫu đối nghịch do 
$\lVert \mathbf{\delta} \rVert_1 = \sum_{i=1}^p |\mathbf{\delta_i}|$ biểu diễn tổng sai khác
của nhiễu và được sử dụng rộng rãi như là hàm thay thế để phát triển các nhiễu thưa. 
Phần đánh giá hiệu năng sẽ chứng minh rằng khi đưa thành phần $L_1$ vào nhiễu sẽ sinh ra 
tập phân biệt các mẫu đối nghịch, và nó cải thiện khả năng chuyển giao tấn công và triển 
khai huấn luyện đối nghịch.

