\section{Thuật toán EAD}
Khi giải bài toán EAD trong \ref{eq:3.5} mà không có thành phần $L_1$, Carlini and Wagner 
sử dụng kĩ thuật COV (change-of-variable) qua hàm \textit{tanh} trên $\mathbf{x}$ nhằm 
mục đích loại bỏ ràng buộc $\mathbf{x} \in [0,1]^p$ (Carlini and Wagner 2017b). Khi 
$\beta > 0$, ta thấy COV sẽ không hiệu quả để giải bài toán \ref{eq:3.5} vì mẫu đối 
nghịch tương ứng sẽ không nhạy cảm với thay đổi trên $\beta$ (xem phần đánh giá hiệu 
năng để biết chi tiết). Vì $L_1$ không khả vi trên toàn không gian, COV không giải 
được bài toán \ref{eq:3.5} do không dùng được phương pháp dưới đạo hàm (\textit{subgradient})
trong việc giải bài toán tối ưu (Duchi and Singer 2009). 

Để giải bài toán EAD trong \ref{eq:3.5} để sinh ra các mẫu đối nghịch, bài viết này đề
xuất sử dụng thuật toán ISTA (\textit{iterative shrinkage-thresholding algorithm – thuật toán lặp 
với ngưỡng biến đổi}) (Beck and Teboulle 2009). ISTA là một thuật toán tối ưu bậc nhất
phổ biến, ISTA thêm một bước điều chỉnh ngưỡng trong mỗi bước lặp. Cụ thể, 
đặt $g(\mathbf{x}) = c \times f(\mathbf{x}) + \lVert \mathbf{x} - \mathbf{x_0} \rVert_2^2$
và đặt $\nabla g(\mathbf{x})$ là gradient của $g(\mathbf{x})$ được tính bởi mạng DNN. 
Tại bước lặp thứ $k+1$, mẫu đối nghịch $\mathbf{x}^{(k+1)}$ của $\mathbf{x_0}$ được tính 
như sau:
\begin{equation}
    \label{eq:3.6}
    \mathbf{x}^{(k+1)} = S_{\beta} (\mathbf{x}^{(k)} - \alpha_k \nabla g(\mathbf{x}^{(k)}))
\end{equation}
Trong đó, $\alpha_k$ là độ dài bước tại bước lặp thứ $k+1$, $S_{\beta} : \mathbb{R}^p \to 
\mathbb{R}^p$ là hàm của phép chiếu biến đổi ngưỡng trên từng phần tử, được xác định bởi:
\begin{equation}
    \label{eq:3.7}
    [S_{\beta}(\mathbf{z})]_i = 
    \begin{cases}
        \min \{ \mathbf{z}_i - \beta, 1 \} &\text{ nếu } \mathbf{z}_i - \mathbf{x_0}_i  > \beta; \\
        \mathbf{x_0}_i &\text{ nếu } |\mathbf{z}_i - \mathbf{x_0}_i| \leq \beta; \\
        \max \{ \mathbf{z}_i + \beta, 0 \} &\text{ nếu } \mathbf{z}_i - \mathbf{x_0}_i < -\beta
    \end{cases}
\end{equation}
Với $i \in \{ 1, ..., p \}$. Nếu $|\mathbf{z}_i - \mathbf{x_0}_i| > \beta$, thành phần 
$\mathbf{z}_i$ được co lại với hệ số $\beta$ và chiếu thành phần kết quả lên miền ràng buộc 
chấp nhận được thuộc đoạn $[0,1]$. Mặt khác, nếu $|\mathbf{z}_i - \mathbf{x_0}_i| \leq \beta$
, nó đặt ngưỡng $\mathbf{z}_i$ bằng cách thiết lập $[S_{\beta}(\mathbf{z})]_i = \mathbf{x_0}_i$.
Chứng minh thuật toán tối ưu sử dụng \ref{eq:3.6} để giải bài toán EAD trong \ref{eq:3.5}
được trình bày trong tài liệu (https://arxiv.org/abs/1709.04114) 

Thuật toán EAD để tạo mẫu đối nghịch được tổng kết trong thuật toán 1. Để tính toán hiệu quả,
 ta triển khai FISTA (fast ISTA) cho EAD, nó sẽ cho tốc độ hội tụ tối ưu với phương pháp 
 tối ưu hóa bậc 1 (Beck and Teboulle 2009). Vector $\mathbf{y}^{(k)}$ trong thuật toán 1 
 kết hợp với momen trong $\mathbf{x}^{(k)}$. Trong phép lặp EAD, bước lặp $\mathbf{y}^{(k)}$ 
 được coi là mẫu đối nghịch thành công của $\mathbf{x_0}$ nếu mô hình dự báo thành lớp 
 đích $t$. Mẫu đối nghịch $\mathbf{x}$ cuối cùng được lựa chọn từ tất cả các mẫu đối nghịch 
 thành công. Trong bài viết này, chúng ta xem xét 2 luật để chọn $\mathbf{x}$ từ tập các 
 mẫu đối nghịch thành công: elastic-net tối thiểu (EN) và nhiễu $L_1$ tương ứng với $\mathbf{x_0}$.
Sự ảnh hưởng của $\beta$, $\kappa$  và luật quyết định lên EAD sẽ được nghiên cứu trong phần tiếp theo.
