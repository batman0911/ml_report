\chapter{Đánh giá hiệu năng}
Trong phần này, chúng ta so sánh EAD với các thuật toán tấn công tiên tiến hiện nay trên 3 
tập dữ liệu ảnh: MNIST, CIFAIR10 và ImageNet. Tác giả bài viết muốn cho thấy:
\begin{itemize}
    \item EAD có thể thu được hiệu năng tấn công tương tự tấn công C\&W đối với mạng DNN 
    phòng thủ và mạng DNN không phòng thủ do C\&W là trường hợp đặc biệt của EAD khi 
    $\beta = 0$.
    \item So sánh với các phương pháp tấn công hiện tại dựa trên $L_1$ như FGM, I-FGM, 
    các mẫu đối nghịch thu được từ EAD có nhiễu $L_1$ nhỏ hơn đáng kể và tỷ lệ tấn công 
    thành công cao hơn.
    \item Các mẫu đối nghịch tạo bởi EAD có thể hỗ trợ chuyển giao tấn công và triển khai 
    huấn luyện đối nghịch.
\end{itemize}
\section{Các phương pháp đối sánh}
Bài viết này so sánh EAD với các phương pháp tấn công nhắm đích hiệu quả nhất có sử dụng 
các loại hiệu chỉnh khác nhau:
\begin{itemize}
    \item C\&W: tấn công nhắm đích dùng hiệu chỉnh $L_2$ được đề xuất bởi Carlini and Wagner (Carlini and Wagner 2017b). Đây là trường hợp đặc biệt của EAD khi $\beta = 0$.
    \item FGM (fast gradient method): được đề xuất bởi (Goodfellow, Shlens, and Szegedy 2015). FGM sử dụng nhiều loại hiệu chỉnh với các phiên bản FGM-$L_1$, FGM-$L_2$, FGM-$L_{\infty}$.
    \item I-FGM: (FGM lặp): được đề xuất bởi (Kurakin, Goodfellow, and Bengio 2016b). I-FGM sử dụng nhiều loại hiệu chỉnh I-FGM-$L_1$, I-FGM-$L_2$, I-FGM-$L_{\infty}$.
\end{itemize}
\section{Thiết kế thực nghiệm}
Thực nghiệm sử dụng framework của Carlini và Wagner \footnote{\href{https://github.com/carlini/nn_robust_attacks}{https://github.com/carlini/nn\_robust\_attacks}}. Với cả tấn công EAD và C\&W, tác giả đều sử dụng các tham số cấu hình mặc định, trong đó triển khai 9 bước tìm kiếm nhị phân trên tham số hiệu chỉnh $c$ (bắt đầu từ 0.001) và chạy  vòng lặp cho mỗi bước với learning rate khởi tạo. Để tìm các mẫu đối nghịch thành công, tác giả sử dụng thuật toán tối ưu tham chiếu (ADAM) cho C\&W, và sử dụng thuật toán chiếu FISTA (thuật toán 1) với square-root decaying learning rate cho EAD. Tương tự tấn công C\&W, mẫu đối nghịch cuối cùng tìm được từ EAD được chọn là mẫu nhiễu ít nhất giữa các mẫu đối nghịch thành công. Độ nhạy cảm của L1 (tham số $\beta$) và tác động của luật chọn trong EAD sẽ được bàn đến ở phần sau. Tác giả đặt tham số chuyển giao tấn công $\kappa = 0$ cho cả EAD và C\&W.


Tác giả triển khai FGM và I-FGM bằng gói thư viện CleverHans \footnote{\href{https://github.com/cleverhans-lab/cleverhans}{https://github.com/cleverhans-lab/cleverhans}}. Tham số nhiễu tốt nhất $\epsilon$ được xác định bằng phương pháp tìm kiếm lưới fine-grained trên mỗi ảnh, $\epsilon$ nhỏ nhất trong lưới đạt được tấn công thành công sẽ được ghi lại. Với I-FGM, tác giả thực hiện $10$ vòng lặp FGM ($10$ là giá trị mặc định) với $\epsilon$-ball clipping. Tham số nhiễu $\epsilon'$ trong mỗi vòng lặp FGM được đặt là $\epsilon/10$ , đã được chứng minh hiệu quả trong (Tramèr et al. 2017). Dải giá trị của lưới và độ phân giải của 2 phương pháp trên được đề cập cụ thể trong tài liệu bổ sung 1.

Việc phân loại ảnh cho tập MNIST và CIFAIR10 được huấn luyện trên mô hình DNN bởi Carlini and Wagner. Để phân loại cho tập ImageNet, tác giả dùng mô hình Inception-v3 (Szegedy et al. 2016). Đối với tập MNIST và CIFAIR10, tác giả lấy ngẫu nhiên trong tập test 1000 ảnh đã được phân loại đúng để tấn công làm phân loại sai. Với tập ImageNet, tác giả chọn ngẫu nhiên 100 ảnh đã được phân loại đúng và 9 lớp sai để tấn công. Tất cả thực nghiệm được triển khai trên thiết bị phần cứng Intel E5-2690 v3 CPU, 40 GB RAM, single NVIDIA K80 GPU. Code thực nghiệm EAD có thể được download từ github \footnote{\href{https://github.com/ysharma1126/EAD-Attack}{https://github.com/ysharma1126/EAD-Attack}}
\section{Đánh giá các độ đo}
Theo cách đánh giá trong (Carlini and Wagner 2017b), tác giải báo cáo tỷ lệ tấn công thành công và độ nhiễu của mẫu đối nghịch trong mỗi phương pháp. Tỷ lệ tấn công thành công (ASR) là phần trăm mẫu đối nghịch được phân loại vào lớp đích (khác lớp gốc). Các trung bình $L_1$, $L_2$ và $L_{\infty}$ của các mẫu đối nghịch thành công cũng được ghi chép báo cáo. Cụ thể, các trường hợp sau được quan tâm:
\begin{itemize}
    \item \textbf{Trường hợp tốt nhất:} Tấn công ít khó nhất trong số các tấn công nhắm đích sử dụng nhiễu.
    \item \textbf{Trường hợp trung bình:} Tấn công nhắm đích vào ngẫu nhiên 1 lớp gán nhãn sai.
    \item \textbf{Trường hợp xấu nhất:} tấn công khó nhất trong số những tấn công nhắm đích sử dụng nhiễu.
\end{itemize}
\section{Phân tích độ nhạy và luật quyết định}
\begin{longtable}[c]{l}
	\resizebox{\textwidth}{!}{
		\begin{tabular}{p{2.5cm}l|llll|llll|llll}
			\hline
			&&\multicolumn{4}{c|}{Best case} & \multicolumn{4}{c|}{Average case} & \multicolumn{4}{c}{Worst case} \\ \hline
			Optimization method & $\beta$ & ASR & $L_1$ & $L_2$ & $L_\infty$ & ASR & $L_1$ & $L_2$ & $L_\infty$ & ASR & $L_1$ & $L_2$ & $L_\infty$ \\ \hline
			\multirow{5}{*}{COV} & 0 & 100 & 13.93 & 1.377 & 0.379 & 100 & 22.46 & 1.972 & 0.514 & 99.8 & 32.3 & 2.639 & 0.663 \\
			& $10^{-5}$ & 100 & 13.92 & 1.377 & 0.379 & 100 & 22.66 & 1.98 & 0.508 & 99.5 & 32.33 & 2.64 & 0.663 \\
			& $10^{-4}$ & 100 & 13.91 & 1.377 & 0.379 & 100 & 23.11 & 2.013 & 0.517 & 100 & 32.32 & 2.639 & 0.664 \\
			& $10^{-3}$ & 100 & 13.8 & 1.377 & 0.381 & 100 & 22.42 & 1.977 & 0.512 & 99.9 & 32.2 & 2.639 & 0.664 \\
			& $10^{-2}$ & 100 & 12.98 & 1.38 & 0.389 & 100 & 22.27 & 2.026 & 0.53 & 99.5 & 31.41 & 2.643 & 0.673 \\ \hline
			\multirow{5}{*}{EAD (EN rule)} & 0 & 100 & 14.04 & 1.369 & 0.376 & 100 & 22.63 & 1.953 & 0.512 & 99.8 & 31.43 & 2.51 & 0.644 \\
			& $10^{-5}$ & 100 & 13.66 & 1.369 & 0.378 & 100 & 22.6 & 1.98 & 0.515 & 99.9 & 30.79 & 2.507 & 0.648 \\
			& $10^{-4}$ & 100 & 12.79 & 1.372 & 0.388 & 100 & 20.98 & 1.951 & 0.521 & 100 & 29.21 & 2.514 & 0.667 \\
			& $10^{-3}$ & 100 & 9.808 & 1.427 & 0.452 & 100 & 17.4 & 2.001 & 0.594 & 100 & 25.52 & 2.582 & 0.748 \\
			& $10^{-2}$ & 100 & 7.271 & 1.718 & 0.674 & 100 & 13.56 & 2.395 & 0.852 & 100 & 20.77 & 3.021 & 0.976 \\ \hline
	\end{tabular}} \\
	\caption[So sánh COV và EAD trên tập MNIST]{So sánh COV và EAD trong việc tìm ra công thức elastic-net trên tập MNIST. ASR là tỷ lệ tấn công thành công. Mặc dù 2 phương pháp trên đều đạt được tỷ lệ tấn công thành công như nhau, COV không hiệu quả trong việc tạo ra các mẫu đối nghịch $L_1$. Khi $\beta$ tăng lên, EAD tạo ra các mẫu đối nghịch $L_1$ ít biến dạng hơn trong khi COV thì không bị ảnh hưởng nhiều khi thay đổi $\beta$.}
	\label{tab:tab_1}
	\\
\end{longtable}

Tác giả kiểm tra sự cần thiết của việc sử dụng thuật toán 1 để giải bài toán tấn công bằng hiệu chỉnh elastic-net trong (\ref{eq:7}) bằng cách so sánh nó với thuật toán COV thuần. Trong tài liệu (Carlini and Wagner 2017b), Carlini và Wagner đã loại bỏ điều kiện ràng buộc $\mathbf{x} \in [0,1]^p$ bằng cách thay $\mathbf{x}$ bằng $\frac{\mathbf{1} + \tanh \mathbf{w}}{2}$ với $\mathbf{w} \in \mathbb{R}^p$ và $\mathbf{1} \in \mathbb{R}^p$ là vector các với các phần tử $1$. Thuật toán tối ưu mặc định ADAM (Kingma and Ba 2014) được sử dụng để giải ra nghiệm $\mathbf{w}$ và từ đó tìm được $\mathbf{x}$. Tác giả áp dụng thuật toán COV lên (\ref{eq:7}) và so sánh với EAD trên tập MNIST với các $\beta$ khác nhau trong Bảng \ref{tab:tab_1} ở trên. Mặc dù COV và EAD thu được tỷ lệ tấn công thành công tương tự nhau, người ta quan sát thấy COV không hiệu quả trong việc tạo ra các mẫu đối nghịch $L_1$. Khi tăng $\beta$, EAD tạo ra mẫu đối nghịch $L_1$ ít nhiễu hơn, trong khi các kết quả $L_1$, $L_2$, và $L_{\infty}$ của COV thì ít chịu ảnh hưởng khi thay đổi $\beta$. Khi sử dụng các hàm thư viện TensorFlow như AdaGrad, RMSProp, SGD, người ta cũng thấy nó ít ảnh hưởng bởi $\beta$ giống như trường hợp của COV. Tác giả cũng chú ý rằng COV không thể dùng phương pháp tối ưu ISTA vì thành phần hàm $\tanh$ trong hiệu chỉnh $L_1$. Sự ít ảnh hưởng của COV cho thấy nó không phù hợp cho tối ưu elastic-net, do nó không hiệu quả cho bài toán tối ưu bằng subgradient (Duchi and Singer 2009). Với EAD, tác giả quan sát thấy sự đánh đổi giữa $L_1$ với 2 độ đo nhiễu còn lại là $L_2$, và $L_{\infty}$. Điều này là do khi tăng $\beta$ làm tăng độ thưa của nhiễu và do đó tăng $L_2$, $L_{\infty}$ . Kết quả tương tự cũng được quan sát thấy với CIFAR10.
\begin{figure}[H] % places figure environment here   
    \centering % Centers Graphic
    \includegraphics[width=1\textwidth]{assets/fig_02.png} 
    \caption{So sánh luật quyết định EN và $L_1$ trong EAD trên tập MNIST với nhiều tham số hiệu chỉnh $L_1$ $\beta$ (trường hợp trung bình). So sánh với luật chọn EN tại cùng giá trị $\beta$ thì luật chọn $L_1$ thu được các mẫu ít nhiễu $L_1$ hơn, nhưng đổi lại có thể bị nhiễu $L_2$, $L_{\infty}$ nhiều hơn.}% Creates caption underneath graph
    \label{fig:fg_02}
\end{figure}

Ở bảng \ref{tab:tab_1}, quá trình thuật toán tối ưu chạy để tìm ra mẫu đối nghịch cuối cùng giữa những mẫu đối nghịch thành công dựa trên hàm mất mát elastic-net trong $\{\mathbf{x}^{(k)}\}^I_{k=1}$, gọi là luật chọn elastic-net. Ngoài ra, có thể chọn mẫu đối nghịch cuối cùng sao cho $L_1$ nhỏ nhất, gọi là luật chọn $L_1$. Hình \ref{fig:fg_02} so sánh ASR và các nhiễu trung bình trên 2 luật chọn nói trên với các giá trị  khác nhau trên tập dữ liệu MNIST. Cả 2 luật chọn đều cho tỷ lệ thành công ASR $100\%$ với dải giá trị $\beta$ rộng. Với cùng 1 giá trị $\beta$, luật chọn $L_1$ cho ra các mẫu đối nghịch có nhiễu $L_1$ nhỏ hơn so với luật chọn EN, trong khi đó lại bị trả giá nhiều nhiễu hơn ở $L_2$, $L_{\infty}$. Xu hướng tương tự cũng được quan sát thấy với dữ liệu CIFAR10 (kết quả đầy đủ với cả 2 tập dữ liệu MNIST và CIFAR có trong tài liệu mở rộng). Trong các thí nghiệm tiếp theo, tác giả báo cáo kết quả của EAD với 2 luật chọn và $\beta = 10^{-3}$, vì trên tập MNIST và CIFAR10 giá trị $\beta$ làm giảm nhiễu $L_1$ trong khi có thể so sánh $L_2$, $L_{\infty}$ với trường hợp $\beta = 0$ (nghĩa là không có hiệu chỉnh $L_1$).


\section{Tỷ lệ tấn công thành công và nhiễu trên các tập dữ liệu MNIST, CIFAR10 và ImageNet}
Nhóm tác giả so sánh EAD với các phương pháp đối sánh trên các tiêu chí tỷ lệ tấn công thành công và các nhiễu khi tấn công mạng DNN huấn luyện bởi MNIST, CIFAR10 và ImageNet. Bảng 2 là kết quả thực nghiệm. FGM ít tấn công thành công (chỉ số ASR thấp) và chỉ số nhiễu khá lớn so với các phương pháp khác. Trong khi đó, C\&W, I-FGM và EAD đều đạt ASR $100\%$. Ngoài ra, EAD, C\&W và I-FGM-$L_{\infty}$ đạt được các mẫu đối nghịch với ít nhiễu nhất lần lượt trên các chỉ số  $L_1$, $L_2$ và $L_{\infty}$. Hơn nữa, EAD tốt hơn các phương pháp $L_1$ hiện tại (ví dụ I-FGM-$L_1$). So sánh với I-FGM-$L_1$, EAD với luật chọn EN giảm nhiễu $L_1$ xuống xấp xỉ $47\%$ trên tập MNIST, $53\%$ với tập CIFAR10 và $87\%$ với ImageNet. Tác giả cũng báo cáo kết quả quan sát rằng EAD với luật chọn $L_1$ có thể giảm nhiễu $L_1$ nhưng làm tăng $L_2$ và $L_{\infty}$.

Mặc dù có nhiễu $L_2$ và $L_{\infty}$ lớn, các mẫu đối nghịch tạo bởi EAD với luật chọn $L_1$ có thể đạt ASR $100\%$ trên tất cả các tập dữ liệu. Nghĩa là nhiễu $L_2$ và $L_{\infty}$ không đủ để đánh giá sức mạnh của mạng neuron. Hơn nữa, kết quả tấn công trong bảng 2 cho thấy EAD có thể thu được 1 tập phân biệt các mẫu đối nghịch khác biệt cơ bản với các mẫu dựa trên $L_2$ và $L_{\infty}$. Tương tự phương pháp C\&W và I-FGM, các mẫu đối nghịch sinh bởi EAD đều khó phân biệt bằng mắt thường. 

\section{Phá vỡ DNN chắt lọc phòng thủ}
Ngoài ra, để tấn công DNN không phòng thủ bằng các mẫu đối nghịch, tác giả đã cho thấy EAD có thể phá vỡ DNN chắt lọc phòng thủ. Chắt lọc phòng thủ (Papernot et al. 2016b) là kĩ thuật phòng thủ chuẩn trong đó mạng được huấn luyện lại bằng các xác suất từng lớp được dự đoán bởi một mạng gốc (chứ không phải bằng nhãn ground truth), gọi là nhãn mềm và người ta dùng tham số nhiệt $T$ trong lớp softmax để tăng cường sức mạnh của nó chống lại nhiễu đối nghịch. Tương tự như phương thức tấn công tiên tiến C\&W, hình \ref{fig:fg_03} cho thấy EAD có thể thu được ASR $100\%$ với các giá trị $T$ khác nhau trên tập MNIST và CIFAR10. Hơn nữa, vì công thức tấn công C\&W là trường hợp đặc biệt của công thức EAD trong (\ref{eq:7}) khi $\beta = 0$, sự thành công của EAD trong việc phá vỡ chắt lọc phòng thủ gợi ý 1 cách mới để tạo ra các mẫu đối nghịch hiệu quả là dùng các tham số  $\beta$ khác nhau cho hiệu chỉnh $L_1$. Kết quả đầy đủ của tấn công ở trong tài liệu mở rộng.

\begin{figure}[H] % places figure environment here   
    \centering % Centers Graphic
    \includegraphics[width=0.8\textwidth]{assets/fig_3.png} 
    \caption{ASR (trường hợp trung bình) của C\&W và EAD trên tập MNIST và CIFAR10 với các tham số nhiệt $T$ khác nhau cho chắt lọc phòng thủ. Cả 2 phương pháp đều phá vỡ thành công chắt lọc phòng thủ.} % Creates caption  % Creates caption underneath graph
    \label{fig:fg_03}
\end{figure}
\section{Cải thiện tấn công chuyển giao}
Trong tài liệu (Carlini and Wagner 2017b) đã chỉ ra rằng tấn công C\&W có khả năng chuyển giao tốt hơn  từ 1 mạng không phòng thủ sang 1 mạng phòng thủ chưng cất bằng cách tối ưu các tham số độc lập $\kappa$ trong (4). Theo (Carlini and Wagner 2017b), nhóm tác giả sử dụng cùng bộ tham số của tấn công chuyển giao trên tập MNIST, vì MNIST là tập dữ liệu khó nhất khi tấn công bằng nhiễu trung bình trên mỗi ảnh như trong bảng 2 ở trên.

Cố định $\kappa$, các mẫu đối nghịch được sinh ra từ mạng không phòng thủ gốc được sử dụng để tấn công mạng phòng thủ chưng cất với tham số nhiệt $T = 100$ (Papernot et al. 2016b). Tỷ lệ tấn công thành công (ASR) của EAD, phương pháp C\&W và I-FGM được trình bày trong hình 4. Khi $\kappa = 0$, tất cả các phương pháp đều cho ASR thấp và do đó không tạo được các mẫu đối nghịch có thể chuyển giao. Tỷ lệ ASR của EAD và phương pháp C\&W cải thiện khi đặt $\kappa > 0$ , trong khi đó tỷ lệ ASR của I-FGM thấp (dưới $2\%$) do tấn công không có các tham số tương tự để có thể chuyển giao.

Chú ý rằng, EAD có thể thu được ASR gần đạt $99\%$ khi $\kappa = 50$, trong khi đó ASR cao nhất mà phương pháp C\&W đạt được là gần $88\%$ khi $\kappa = 40$. Chứng tỏ rằng, khả năng chuyển giao tấn công tăng lên khi sử dụng các mẫu đối nghịch được tạo từ EAD, điều này là do thuật toán ISTA trong (8) mạnh hơn so với tấn công C\&W qua phương pháp shrinking and thresholding. Nhóm tác giả cũng phát hiện rằng khi đặt $\kappa$ quá lớn cũng làm giảm ASR của các tấn công chuyển giao cả bằng EAD và phương pháp C\&W, do thuật toán tối ưu không tìm được mẫu đối nghịch có thể làm tối thiểu hóa hàm mất mát $f$ trong (4) khi $\kappa$ lớn. Kết quả đầy đủ của việc chuyển giao tấn công được báo cáo trong tài liệu kèm theo.

\begin{figure}[H] % places figure environment here   
    \centering % Centers Graphic
    \includegraphics[width=0.5\textwidth]{assets/fig_04.png} 
    \caption{Khả năng chuyển giao tham số $\kappa$} % Creates caption  % Creates caption underneath graph
    \label{fig:fg_04}
\end{figure}
\section{Huấn luyện đối nghịch bổ sung}
Để xem xét sự khác biệt giữa các mẫu đối nghịch $L_1$ và $L_2$, tác giả kiểm tra hiệu quả huấn luyện đối nghịch trên tập MNIST. Họ chọn ngẫu nhiên $1000$ ảnh từ tập huấn luyện và sử dụng tấn công C\&W và EAD (luật chọn $L_1$) để tạo ra $9000$ mẫu đối nghịch cho tất cả các nhãn sai với mỗi phương pháp. Sau đó, họ thêm vào tập huấn luyện các mẫu đối nghịch này để huấn luyện lại mạng và kiểm tra sức mạnh của nó trên tập kiếm thử, kết quả tổng hợp trong bảng 3. Để huấn luyện đối nghịch với 1 phương pháp nào đó, mặc dù cả 2 tấn công đều đạt tỷ lệ thành công trung bình $100\%$, mạng vẫn có sức chịu đựng tốt hơn trước nhiễu đối nghịch vì các chỉ số nhiễu đều đã tăng lên đáng kể so với trường hợp chưa huấn luyện. Tác giả cũng quan sát thấy kết hợp huấn luyện bằng EAD và phương pháp C\&W có thể làm tăng hơn nữa nhiễu $L_1$ và $L_2$ so với tấn công C\&W, và tăng nhiễu $L_2$ so với EAD, với giả thiết rằng mẫu $L_1$ được tạo bởi EAD có thể được huấn luyện đối nghịch bổ sung.  
